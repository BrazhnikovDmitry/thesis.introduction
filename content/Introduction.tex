%\documentclass[12pt]{article}
%\usepackage[margin=1 in, head=0.9 in]{geometry}
%\usepackage{fancyhdr}
%\usepackage{listings}
%\usepackage{caption}
%\usepackage{color}
%\usepackage{xcolor}
%\usepackage{caption, apacite}
%\DeclareCaptionFont{white}{\color{white}}
%\DeclareCaptionFormat{listing}{\colorbox{gray}{\parbox{\textwidth}{#1#2#3}}}
%\captionsetup[lstlisting]{format=listing,labelfont=white,textfont=white}
%\usepackage{graphicx}
%\usepackage{amsmath, amssymb, amsthm}
%\usepackage[all,cmtip]{xy}
%\pagestyle{fancy}
\input{/home/dmitry/Work/Research/thesis/FINALE/settings.tex}

%\newcommand{\ib}[1]{\bar{#1}}
%\newcommand{\ip}[1]{#1^{\prime}}
%\newcommand{\iti}[1]{\tilde{#1}}
%\newcommand{\pder}[2][]{\frac{\partial #1}{\partial #2}}
%\newcommand{\pderr}[3][]{\frac{\partial^2 #1}{\partial #2 \partial #3}}
%\newcommand{\ap}[1]{\langle #1 \rangle}
%\newcommand{\hx}[2][]{ \hat{#1}_{#2} }
%\newcommand{\hxd}[3][]{ {\hat{#1}_{{#2}_{#3}}} }
%\newcommand{\mb}[1]{\mathbf{#1}}
%\newcommand{\pp}[1]{\partial_{#1}}
%\newcommand{\vphi}{\varphi}
%\newcommand{\hqu}{\mathcal{H}}
%\newcommand{\cj}[1]{{#1}^{\star}}
%\newcommand{\rc}{\operatorname{Re}}	% complex real
%\newcommand{\ic}{\operatorname{Im}}	% complex real
\begin{document}

\title{Gravity wave scattering by large scale sea bottom features}
\maketitle

\section{Abstract}
Oceanic gravity waves fundamentally contribute to redistribution of mechanical energy in the world ocean. Energy transfer from wave generation sites to dissipation regions are controlled by the physics of gravity waves. The energy transfer occurs not only in the lateral direction but also by conversion into vertical movements such as internal waves. For surface and interior fluid motions, interaction with the sea bottom irregularities can alter gravity wave energy pathways. That is energy can be reflected and scattered both in different directions and into propagation modes. My PhD project addresses wave scattering and reflection by sea bottom features in two important cases: a tsunami wave scattered from a prominent sea mount and reflection of a semidiurnal internal tide from a continental slope.\\
Evidently, bottom bathymetry determines local wave behaviour both for barotropic and baroclinic modes. The gravity wave scattering and excitation of subsidiary wave patterns controls energy pathways on the world ocean scales. This thesis intends to investigate physical aspects of wave energy scattering, mode conversion over topographic features and variability in these process.\\
Interactions of tsunami waves with seafloor topography can delay and redirect the energy flux, \textit{creating amplified waves with long periods}. During the March 2011 tsunami event in Japan tide stations in Northern and Central California recorded large waves, while much smaller waves were observed in Washington and in the southern California. This correlated with a ray of high energy originating from the distant scattering from Emperor seamount chain and more specifically from Koko Guyot. This underwater feature focuses energy into tight beams. This investigation shows that guoyt's shape is responsible for the observed signals. The 2011 tsunami event is compared with a case of Kuril tsunami of 2006. The comparative analysis emphasis strong correlation between frequency of incident tsunami wave component, its propagation direction and the observed signal at Northern California.\\
The energy can be distributed not only by surface expression of the sea but also in the ocean's interior. Internal tides originate as surface tide encounters steep topographic feature. At such location the energy would drain into perturbations of isopycnals that further can travel hundreds of  leagues without any significant loss of energy. In the second part of my thesis origination, propagation and dissipation of the internal tidal beam is investigated on the example of Tasman Sea. 
A global internal tide model, satellite altimetry measurements and in situ observations indicate that a ridge south of New Zealand is a site for the generation of the strong tidal beam that is directed towards the Tasman Shelf break. The low mode internal wave beam undergoes scattering from topography, which transfers energy to shorter length scales. Thus, understanding of local energy processes will lead to a more detailed picture of tidal energy dissipation.\\
\textbf{CONCLUSIONS}

\section{Introduction}
Gravity waves are ubiquotous feature of the World Ocean. The fundamental importance of the linear gravity waves is their ability to transfer momentum over large length scales. The energy pathways of propagating gravity waves can be altered in a result of interaction with large scale sea bottom features such as submarine seamountains, trenches and continental shelves. These processes appreciably shape wave propagation, and thus, leading to variation in localized energy budgets. In this work two geophysically important cases are considered. Two examples of energy scattering will be considered in this thesis work.\\
The first deals with tsunami waves (\textbf{HERE IT GOES LONG INTRODUCTION to TSUNAMI WAVE: IMPORTANCE, WHERE IT OCCURS, WHAT IT CAUSES, SECTION?})\\
What is missing? References. More generalization? Decay? Murty, Rabinovich, Kowalik, that Greek lady.
Tsunami waves are transient waves occurring owing to its generation earthquakes along deep ocean trenches (impulsively forced by submarine earthquakes). Rapid uplift of sea bottom creates a hump water that further freely propagates away from the trench in deep ocean. During their transoceanic propagation these waves encounter numerous large scale ocean bottom roughness such as seamount chains and submarine plateaus. All of these obstacles can redirect tsunami propagation because of rapid change in ocean depth. This phenomena is well known as wave refraction in coastal oceanography.\\
For the US it is of most importance understanding how tsunami waves are transformed when their origination lays along Japan-Kuril trench (Figure 1 map. There one of the most disruptive and powerful events in the recorded history occurred. For example, tsunami event happened near Japan on March 12, 2011 created a devasting wave that caused huge along shore run ups in Japan, the wave crossed the Northern Pacific Ocean and caused severe economical losses to the West Coast. In this example it is emphasized that interaction with prolonged chain of seamounts known Emperor Seamount chain has created additional conditions for relatively directive and focused signal on the West Coast.\\
Several previous studies have identified that the primary source of scattering of tsunami waves in Pacific Ocean is Koko Guoyt. Here the secondary waves originate from interaction with steep seamount. Later secondary waves form a distinctive beam which has its final destination towards Mendocino Escarpment where additional focusing occurs bringing amplified waves to Crescent City, CA. In such scenario it is notable that observations suggest arrival of the secondary waves long after the main tsunami wave arrival occurs. Hence, developing understanding of mechanisms behind generation of the scattered tsunami waves have important forecasting consequences. This work is developing a description for such mechanism.\\

\begin{figure}
\includegraphics[scale=0.5]{../figures/map_w_places.pdf}
\caption{The map of the Pacific Ocean with outlined regions.}
\end{figure}
(\textbf{HERE IT GOES LONG INTRODUCTION to INTERNAL TIDES, GENERATION, GLOBAL SCALE, IMPORTANCE, SECTION?})\\
Another type of oscillatory motions occurring in the World Ocean is internal waves. Their existence owns to stratification of the water column. Disturbance of isopycnal surfaces create perturbations which can than freely propagate from generation region over large distances. The type of sourcing will define kinematic properties of the generated internal waves. And there is a large variety of generation mechanisms: ice keels, wind forcing, surface buoyancy forcing. Here I deal with the wave motions created by oscillatory stratified tidal flow hence making existence of internal tides. Internal tides are internal waves of tidal or quasi-tidal frequency. Only recently it became clear that internal tides can transfer energy from steep topogrpahic features crossing the World Ocean and depositing their momentum creating upward motion of water column properties. Internal tides are believed to be an agent in shaping the Meridional overturning circulation.\\
Internal tides propagation characteristic and scattering properties are still poorly understand and quantified. This is a goal for Tasman Tidal Dissipation Experiment (TTIDE). Their phenomena to investigate was a scattering of internal tidal beam that crosses Tasman Sea and impinges on continental slope of Tasmania (Figure 1). The primary problem is quantifying how much energy is being reflected into the open ocean. This problem was studied by means of numerous numerical experiments. One of the major concerns presented here is a variability associated with incidence of the tidal beam, intrinsic interactions with topography and background conditions. The description of this mechanisms are useful for description of field observations and consequently creating large scale understanding of internal tide reflection which allows for more describing effects of internal tides on the World Ocean.\\
\textbf{PLUGIN: 2 TW, 1 TW, Macquarie Ridge, internal tides produce transport of particulates: sediments and can cause strong currents in deep ocean. Role in tidal dissipation, Internal tides what became apparent recently are important energy transporters.}
Both of the investigated geophysical problem represent a type of wave scattering by inhomogeneity in the media. The latter here means rapid change in ocean depth that creating conditions for wave's propagation and energetic characteristics change. As title of thesis implies the problems are dealt by application of energy considerations, mainly energy flux.

\subsection{Energy conservation and energy flux}
Even though tsunami being surface waves and internal tides are waves traveling on isopycnal surfaces they posses a similarity which enables similar treatment of their energy properties. For both cases the waves are predominantly long, their typical wavelength is much larger than the ocean's mean depth with periods longer than buoyancy period. Both tsunami and internal tides have scales of 200 km. This provides unified description by means of Laplace tidal equations on f-plane with allowance for density stratification (\cite{kundu2008fluid}, \cite{cushman2011introduction}) (or Shallow Gravity waves) which formulated as follows
\begin{align}
\pder[u]{t} - f v = -\frac{1}{\ib{\rho}_0}\pder[p]{x}\\
\pder[v]{t} + f u = -\frac{1}{\ib{\rho}_0}\pder[p]{y}\\
0 = -\pder[p]{z} - \rho g\\
N^2 w = -\frac{1}{\ib{\rho}_0} \pderr[p]{z}{t}\\
\pder[u]{x} + \pder[v]{y} + \pder[w]{z} = 0
\end{align}
with $(u,~v,~w)$ being velocity along zonal, meridional and vertical axis, $f$ - Coriolis parameter, $p$ - perturbation pressure rising from sea level or isopycnal motions of stratified water column given by Brunt-Vaisala frequency $N^2 = -\frac{g}{\ib{\rho}_0}\pder[\rho_0]{z}$. The set of equations is constructed for linear and hydrostatic flow under Boussinesq approximation. Additionally, boundary conditions are imposed on the surface (linearized) and flat bottom,
\begin{align}
w|_{z = 0} = \pder[\zeta]{t},~p|_{z = 0} = \ib{\rho}_0 g \zeta\\
w|_{z = -H} = 0
\end{align}
Such formulated problem allows separation of vertical axis by employing vertical mode decomposition,
\begin{align}
(u, v, p)(x,y,z,t) = \sum_{n = 0} [u_n(x,y,t), v_n(x,y,t), p_n(x,y,t)]\psi_n(z)\\
w(x,y,z,t) = \sum_{n = 0} [w_n(x,y,t)] \int_{-H}^z \psi_n(z) dz\\
\rho(x,y,z,t) = \sum_{n = 0} [\rho_n(x,y,t)] \frac{d \psi_n(z)}{dz}
\end{align}
Upon substitution of the decompositions Sturm-Liouville problem is obtained for vertical structure (basis) functions $\psi_n(z)$,
%\pder[\frac{1}{N^2 - \omega^2} \pder[\psi_n]{z}]{z} + \big(1 - \frac{f^2}{\omega^2} \big) \frac{\psi_n}{c^2_n} = 0
\begin{equation}
\frac{d}{dz}\big( \frac{1}{N^2} \frac{d \psi_n}{dz} \big) + \frac{1}{c^2_n}\psi_n = 0
\end{equation}
with $c_n$ being an eigenvalue for a respective $n$-th mode.\\
The carried out vertical decomposition effectively decouples vertical dynamics from horizontal reducing the initial set to shallow water-type of equations,
\begin{align}
\pder[\vec{u}_n]{t} + f \vec{k} \times \vec{u}_n = -\frac{1}{\rho_0} \nabla p_n\\
\frac{1}{\rho_0} \pder[p_n]{t} + g D_n \nabla  \cdot \vec{u}_n = 0
\end{align}
Here I intentionally introduced equivalent depth $D_n,~c_n = \sqrt{g D_n}$. For typical oceanic conditions $0$-th mode represents a barotropic solution with $\psi_0 = const$, i.e.  vertically uniform dynamics, and $D_0 \simeq H$ (\cite{hendershott1981long}. So it resembles pure surface wave propagation such as tsunami wave propagation. The left out infinite sequence of vertical modes are purely internal modes that cause isopycnal displacements of magnitude, $\eta_n = \int w dt = \int w_n \int_H^z \psi_n dz$ and negligibly small sea level motions. The latter allows complete separation between barotropic and baroclinic (vertically dependent) motions.\\
For tsunami waves due to their high frequency rotational effects are usually neglected, but can appear over long distances (Kowalik, Sumatra). Internal tides on the other hand are Sverdrup waves which are the same as long shallow water waves but perturbed with rotation causing them to be dispersive.\\
This set enables to provide a plane wave solution described with
\begin{align}
\label{In:svw}
\begin{split}
p_n = p_{0n} e^{i \omega t - \vec{k} \cdot \vec{x}}\\
u_n = \frac{1}{\rho_0} \frac{-\omega k_x + i f k_y}{\omega^2 - f^2} p_n\\
v_n = \frac{1}{\rho_0} \frac{-\omega k_y - i f k_x}{\omega^2 - f^2} p_n
\end{split}
\end{align}
so that wave associated current is related to pressure by polarization relations.\\
Now using framework (12-13) energy equation can be formed by multiplying momentum equations with $\vec{u}_n$ and continuity equation with $p_n$ adding both expressions and depth integrating (\cite{nekrasov1990energy}, \cite{kowalik2013oceanography}, \cite{kelly2012cascade}),
\begin{equation}
(E_k + E_p)_t + \nabla \cdot \vec{F} = D \label{In:eneq}
\end{equation}
Additionally, depth averaging is carried out so that energy flux will be defined,
\begin{equation}
\vec{F}_n = \frac{1}{H} \vec{u_n}^(t) p_n(t) \int_{-H}^0 \psi^2(z) dz \label{In:fldef}
\end{equation}
For tsunami wave propagation due to their barotropic phenomena $\psi(z) = 1$ and $p_n = \rho_0 g \eta$, the energy flux comes to its classical form for surface wave propagation, \cite{kowalik2013oceanography}, $\vec{F} = \rho_0 g \vec{u} \zeta$. I deliberately abandon nonlinearity terms presuming deep ocean propagation. Koko Guoyt depth on average is around 500 m, barotropic velocity is of order $0.1~m/s$, nonlinear terms scale as $\sim u^3$, while linear energy flux as $\sim g u \zeta$. It suggests that only in shallow water where horizontal acceleration is of order of gravity acceleration, nonlinear terms will be a significant addition to total energy transports.\\
In internal tidal studies I additionally carry out period averaging due to harmonic character of waves which is given by Euler formula, $(u,v,p) \sim e^{i \omega t}$, so that (\ref{In:fldef}) becomes
\begin{align}
\vec{F}_n = \frac{1}{2} \cj{\vec{u}}_n p_n \int_{-H}^0 \psi^2(z) dz
\end{align}
The same discussion on nonlinear additions stays true for internal tide scattering from the Tasman continental slope which depth does not exceed $100~m$. Only on shallow shelfs advection of kinetic energy becomes non-negligible (\cite{kang2012energetics}, \cite{nash2012unpredictable}).\\
Energy conservation, \ref{In:eneq} sets an useful framework for analysis of the wave interaction with topography. As bottom slope starts rapidly change, wave propagation undergoes variation resulting in reorientation of currents, divergence of water column. But because of energy conservation (under assumption of nonfrictional flow), amount of total energy should be preserved. Hence, transports of energy through control volumes can present a valuable tool for description of wave interaction with bottom relief. The assumption of nonfrictional flow does not fully hold in the problems considered, but dissipation is not considered in this thesis because of its not primary reason. So that any inconsistencies in control volume calculations are said to be due to either dissipation (both physical and numerical) and due to errors in analysis.\\
After normal mode framework was formulated it is instructional to consider how well it is hold. For tsunami wave this check can be easily done by comparison of Airy wave dispersion (\cite{kundu2008fluid}) with long wave approximation (Figure 2). It is of no surprise that on top Koko Guoyt and on its walls the tsunami break limit for shallow water approximation, so that assumption on flat bottom breaks and validity of the presented dynamics is under suspicion.\\
For internal tides application of normal mode analysis is invalid since for sloping bottom full threedimensional structure should be considered. This would lead to a hyperbolic equation which lands a wave propagation along characteristics (\cite{sandstrom1969effect}) or internal wave beams. their direction are set by the buoyancy frequency,
\begin{equation}
\gamma = \frac{dz}{dx} = \pm \frac{\omega^2 - f^2}{N^2 - \omega^2}
\end{equation}
In case of bottom slope exceeding characteristics angle, $\nabla H > \gamma$ internal waves are reflected (\textit{supercritical regime}), while less inclined topographies there is a forward propagation (\textit{subcritical regime}). The analytical solution for internal wave propagation on sloping bottom was worked out previously (\cite{wunsch1968propagation}, \cite{wunsch1969progressive}) can be compared with normal mode fitting (Figure 3). This  establishes that for internal tide interaction with supercritical continental slope the normal mode approach fails. 

\begin{figure}
\mfig[0.5]{tsunami_regimes.pdf}
\caption{Tsunami regime propagation. Shallow water limit vs deep water limit}
%\includegraphics[scale=0.5]{../}
\end{figure}

\begin{figure}
\mfig[0.8]{errors2.pdf}
\caption{Internal wave propagation on slope in nonrotating, uniformly stratified ocean with inclined bottom.}
%\includegraphics[scale=0.5]{../}
\end{figure}
Nevertheless away from inclined topography normal modes is a successful approach in understanding how wave energetics changes upon interaction with the prominent bottom relief. So what actually happens when the wave encounters bottom slope? Wave scattering.

\subsection{Formulation of scattering problem}
In the previous section the normal mode approach was formulated under condition of flat bottom. This produces a simplified boundary condition $w|_{z = -H} = 0$ which further enables variable separation. Thence, vertical motions became decoupled from the horizontal variability. As wave encounters sloping bottom the boundary condition does not hold any more and becomes fully nonlinear,
\begin{equation}
\vec{u} \cdot \nabla h = 0 \label{In:bcnon}
\end{equation}
Now the boundary condition clearly states that there is no mass transport across impermeable bottom as wave impinges. Wave itself cannot satisfy (\ref{In:bcnon}), hence, additional motions should arise such that the total mass transport will be zero across the boundary. In the most simplistic configuration when a wall is placed against incident wave train, a specular reflection will occur in order to satisfy (\ref{In:bcnon}). Hence, the reflected wave is ``generated" or this can be restated as an incident wave was scattered into a reflected one. To additionally emphasize this point, here and forth wave scattering is a process of wave generation in order to satisfy of no mass transport across ocean's bottom.\\
By now considering a step-like discontinuity in ocean depth the so-called matching conditions can be devised (\cite{mei1989theory}) from (\ref{In:bcnon}) and continuity equation for scattering of surface gravity waves,
\begin{align*}
\vec{u}_1 h_1 = \vec{u}_2 h_2,~\zeta_1 = \zeta_2
\end{align*}
which restates the previous point and additionally emphasizes continuity of sea level. The same approach is also used in studies of internal tide scattering (\cite{larsen1969internal}, \cite{chapman1981scattering}) and internal tide generation problems (\cite{st2002role}). But with a separated condition of no flow through a barrier. In actuality, internal tide production can be viewed as scattering of surface, barotropic tide by steep bottom relief (\cite{hendershott1981long}). Internal tide scattering produce not only change of direction such as for surface waves, but also energy leakage into other vertical modes. MORE ON INTERNAL TIDE SCATTERING Griffith\& Grimshaw.\\
Of additional note, in rotating ocean the boundary condition has the same form, but physically, due to elliptic particle motions in Sverdrup wave \eqref{In:svw}, a coupling occurs between zonal and meridional current components \cite{greenspan1968theory}.\\
From energetic point of view as wave scattering occurs portion of the inicident energy is transfered into other wave motions. For tsunami wave this can be characterized by differential scattering cross section (\cite{landau1988hydrodynamics}) defined as an amount of energy emerged in some differential direction $(\theta, \theta + d \theta)$.\\
In internal tide studies it is of more importance to indentify energy transports from the incident field into other modes (\cite{kurapov2003m}, \cite{kelly2012cascade}. Such that
\begin{align}
C_{n}^{bt-bc} = w_{bt} \cdot p_n\\
C_{mn}^{bc-bc} = (\cj{\vec{u}}_m p_n \int_{-H}^0 \psi_m \nabla \psi_n dz - \cj{\vec{u}}_n p_m \int_{-H}^0 \psi_n \nabla \psi_m dz)
\end{align}
In these two expressions it is explicitly made a separation between transfer from baroptropic tide into internal tidal mode $n$. And in the second energy transfer from internal tidal mode-$n$ to mode-$m$.\\
As scattering occurs, no one can guarantee that incident wave will preserve its form and its energy. So that, for example, a barotropic Sverdrup wave will transfer partially its energy into Kelvin wave (\cite{pinsent1972kelvin}). Or that scattered waves will be free. Trapping of wave energy is a well known phenomena (\cite{longuet1967trapping}, \cite{kowalik2002tidal}).\\
Since in this work energy approached is considered, it is advisable to formulate scattering problem in terms of wave energetics. Aforementioned, scattered waves are generated from the incident wave field, so there is a direct energy transfer which is controlled by \eqref{In:bcnon}. Let consider a system of reference moving with an incident wave. Than the boundary condition will simply postulate of scattered field generation. This can be transferred into an energetics. The boundary surface will hence doing of work on scattered field. So that emitted energy flux is given by,
\begin{equation}
F_{sc} = p_{sc} u_{sc}
\end{equation}
It can be shown (\cite{morse1958theoretical})	that it will have angular distribution in far from the scatter, so differential scattering cross-section can thus introduced.\\
If angular distribution 
 \\

Generally speaking, wave scattering happens due to inconsistency of boundary conditions. This can be expressed as the following equation.\\

Internal tide generation was ascribed to scattering of barotropic tidal energy from bottom relief.

\newpage
\section{TO DO}
\begin{itemize}
\item In my writing here I need to write down energy conservation equation. It must be done super careful. Ref: Wunsch, Ferrari; Nekrasov; 
\item Definition of energy flux since it is corner stone for the whole work. Ref: Kowalik papers; Nash, Kelly

\end{itemize}

\bibliographystyle{apacite}
\bibliography{/home/dmitry/Bibtex_lib/my_first_lib}

\end{document}