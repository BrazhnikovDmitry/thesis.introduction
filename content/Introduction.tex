%\documentclass[12pt]{article}
%\usepackage[margin=1 in, head=0.9 in]{geometry}
%\usepackage{fancyhdr}
%\usepackage{listings}
%\usepackage{caption}
%\usepackage{color}
%\usepackage{xcolor}
%\usepackage{caption, apacite}
%\DeclareCaptionFont{white}{\color{white}}
%\DeclareCaptionFormat{listing}{\colorbox{gray}{\parbox{\textwidth}{#1#2#3}}}
%\captionsetup[lstlisting]{format=listing,labelfont=white,textfont=white}
%\usepackage{graphicx}
%\usepackage{amsmath, amssymb, amsthm}
%\usepackage[all,cmtip]{xy}
%\pagestyle{fancy}
\input{/home/dmitry/Work/Research/thesis/FINALE/settings.tex}

\begin{document}

\title{Gravity wave scattering by large scale sea bottom features}
\maketitle

\section{Abstract}
Oceanic gravity waves fundamentally contribute to redistribution of mechanical energy in the world ocean. Energy transfer from wave generation sites to dissipation regions are controlled by the physics of gravity waves. The energy transfer occurs not only in the lateral direction but also by conversion into vertical movements such as internal waves. For surface and interior fluid motions, interaction with the sea bottom irregularities can alter gravity wave energy pathways. That is energy can be reflected and scattered both in different directions and into propagation modes. My PhD project addresses wave scattering and reflection by sea bottom features in two important cases: a tsunami wave scattered from a prominent sea mount and reflection of a semidiurnal internal tide from a continental shelf break.\\
Interactions of tsunami waves with seafloor topography can delay and redirect the energy flux, creating amplified waves with long periods. During the March 2011 tsunami event in Japan tide stations in Northern and Central California recorded large waves, while much smaller waves were observed in Washington and in the southern California. This correlated with a ray of high energy originating from the distant scattering from Emperor seamount chain and more specifically from the Koko Guyot. The seamount resonance frequency is believed to determine the local tsunami wave dynamics such that energy scattering led to an amplified signal.\\
The energy can be not only redirected but also converted into different vertical modes. Such transformation of the internal tides occurs over continental shelf breaks. The sharp topography causes reflection, mode conversion and local dissipation of various wave field components. A global internal tide model, satellite altimetry measurements and in situ observations indicate that a ridge south of New Zealand is a site for the generation of strong internal tide beam that is directed towards the Tasman Shelf break. The low mode internal wave beam undergoes scattering from topography, which transfers energy to shorter length scales. Thus, understanding of local energy processes will lead to a more detailed picture of tidal energy dissipation.\\
Evidently, bottom bathymetry determines local wave behaviour both for barotropic and baroclinic modes. The gravity wave scattering and excitation of subsidiary wave patterns controls energy pathways on the world ocean scales. This thesis intends to investigate physical aspects of wave energy scattering,  mode conversion over topographic features and variability in these process.

\section{Introduction}
Gravity waves are ubiquotous feature of the World Ocean. The fundamental importance of the linear gravity waves is their ability to transfer momentum over large length scales. The energy pathways of propagating gravity waves can be altered in a result of interaction with large scale sea bottom features such as submarine seamountains, trenches and continental shelves. These processes appreciably shape wave propagation, and thus, leading to variation in localized energy budgets.\\
Two examples of energy scattering will be considered in this thesis work. The first deals with tsunami waves (\textbf{HERE IT GOES LONG INTRODUCTION to TSUNAMI WAVE: IMPORTANCE, WHERE IT OCCURS, WHAT IT CAUSES, SECTION?})\\
Tsunami waves propagate 1000s of kilometers without almost any dissipation of their energy. At their arrival at distant locations such as shorelines the carried mechanical energy leads to devasting events: strong currents and high runups.\\
(\textbf{HERE IT GOES LONG INTRODUCTION to INTERNAL TIDES, GENERATION, GLOBAL SCALE, IMPORTANCE, SECTION?})\\
Another example of energy transfer over long distances is a baroclinic waves of tidal frequencies (internal tides). Their origin usually occurs at long ridge chains. As they cross ocean basins they as well preserve much of their initial energy content which further lead to deposition of the energy.\\


\bibliographystyle{apacite}
\bibliography{/home/dmitry/Bibtex_lib/}

\end{document}